\documentclass [11pt]{article}
\usepackage{graphicx} % Required for inserting images
\usepackage[left=2cm,right=2cm,top=0.5cm,bottom=2cm]{geometry}
\usepackage[T1]{fontenc}
\usepackage[utf8]{inputenc}
\usepackage[french]{babel}

\title{Projet: BEtag}
\author{Noah Moussaoui, Alistaire Vionne, Kaan Akman}
\date{8 Décembre 2023}

\begin{document}
\maketitle
\begin{figure}[h]
    \centering
    \includegraphics[scale=0.4]{institution_v3.png}
    \label{ucl}
\end{figure}

\section{Introduction}
Nous sommes le groupe P3A situé à Charleroi sur le campus de l’UCLouvain. Nous nous sommes
engagés à travailler sur un projet qui visait, dans un premier temps, à travailler en groupe, renforcer nos
liens et mettre en commun nos connaissances. Nous avons pu prendre connaissance sur l'UI (User Interface), la cryptographie et les appliquer.

\section{Contexte}
Notre groupe a été recrutée au service d’aide à l’enfance de l’état belge afin de pouvoir suivre l’état d’éveil du bébé et de la quantité de lait consommé par celui-ci. Il nous est demandé de créer un outil permettant de gérer ces fonctions pour permettre l’amélioration de la santé du bébé. Cet outil permettra de communiquer l’état d’éveil du bébé d’un be\string:bi à l’autre et d’enregistrer la quantité de lait consommé pendant la journée. Finalement, ces 2 be\string:bi doivent être capable d’être dans un même canal et de garder ces messages de façon sécurisés.
\newpage
\section{Fonctionnement du BEtag}
\noindent Voici le fonctionnement de notre be\string:bi:
\begin{enumerate}
    \item Tout d’abord, nous allumons les 2 be\string:bi (parent et enfant) et nous établissons la connexion entre ces 2 be\string:bi via un key (‘Betag’) commun qui permettra de garder la connexion entre-nous et empêcher à des tiers d’y avoir accès à notre réseau radio. Ce key est créé de la façon suivante:
be\string:bi Parent choisit un mot de passe (P) qui sera crypté en Vigenère avec un générateur de nombre aléatoire (A). Le format de notre message est la suivante: .Ensuite, P sera décrypté par le be\string:bi Enfant qui va ensuite utiliser une fonction déterministe F pour calculer F(A). Ce F(A), dans notre code, sera mis en ‘string’ et finalement subira un hashing.
    
\noindent Ce hashing sera ensuite rapporté au be\string:bi parent, entre-temps le be:bi parent aura fait un hashing aussi pour le calcul de F(A). Finalement, ce be:bi parent compare le hashing du be\string:bi parent et du be\string:bi bébé. Si ces deux hashing sont identiques, alors elle définit un key commun avec le be\string:bi bébé.

\noindent Ajouter les points forts et contraintes aussi
    
    \item Après avoir établi la connexion entre les 2 be\string:bi avec un même key, nous allons pouvoir intéragir avec le menu intéractif du be\string:bi parent.

\end{enumerate}

\section{Conclusion}
Dire qu'on a fait les fonctionnalités de base + implémenter des fonctions supp


\end{document}
